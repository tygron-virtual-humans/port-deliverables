\section{Product Backlog}
In this section you can find the product backlog for our part of the project. It is made using the MoSCoW method. This method consists of four sections; Must have, Should have, Could have, Won't have. The user stories can be split into these sections to get a good overview of the important and less important aspects of the product.

\subsection{User stories of features}
\paragraph{Must have} 
\begin{itemize}
\item There needs to be a fully functional port for \gls{GAMYGDALA} to Java.
\item The \gls{GAMYGDALA} plugin needs to work in \gls{GOAL}.
\item The plugin must work without fatal problems, like frozen screens of the program closing itself.
\end{itemize}

\paragraph{Should have} 
\begin{itemize}
\item The plugin should work efficient, so that the programmer will not notice the calculation time.
\item The plugin should work without any errors or failures. 
\end{itemize}

\paragraph{Could have} 
\begin{itemize}
\item There could be a plugin framework for \gls{GOAL}, so that new plugins can easily be added.
\end{itemize}

\paragraph{Won't have} 
\begin{itemize}
\item There will not be new customizable emotions in \gls{GAMYGDALA} on initialization of the plugin.
\item There will not be automatic usage of the plugin. The programmer needs to make a call to the \gls{GAMYGDALA} plugin within is \gls{GOAL} code.
\end{itemize}

\subsection{User stories of technical improvements}
\paragraph{Must have} 
\begin{itemize}
\item The code must be implemented according to the \gls{SEM}.
\item The code must have a test percentage of at least 70\%.
\item There must be unit tests as well as integration tests.
\end{itemize}

\paragraph{Should have} 
\begin{itemize}
\item There should be proper Javadoc for the code.
\item The code should pass the \gls{Checkstyle} demands.
\end{itemize}

\paragraph{Could have} 
\begin{itemize}
\item There could be a higher overall test percentage.
\end{itemize}

\paragraph{Won't have} 
\begin{itemize}
\item There won't be an improvement over the standard \gls{GOAL} framework.
\item The functionality of \gls{GAMYGDALA} will not be extended.
\end{itemize}

\subsection{User stories of know-how acquisition}
\paragraph{Must have} 
\begin{itemize}
\item There must be a proper understanding of the \gls{GOAL} code.
\item There must be a proper understanding of the \gls{GAMYGDALA} code.
\item We must obtain and use the proper \gls{SCRUM} skills.
\item We must obtain and finish the informationskills section.
\item We must obtain knowledge about interactive de\gls{SIG}n.
\end{itemize}

\paragraph{Should have} 
\begin{itemize}
\item We should have a global understanding of the other parts of this project.
\item We should have an understanding of the \gls{GOAL} programming language.
\item We should have a good understanding of Git/GitHub.
\end{itemize}

\paragraph{Could have} 
\begin{itemize}
\item We could obtain knowledge about EIS.
\item We could obtain general knowledge about emotions in AI.
\end{itemize}

\paragraph{Won't have} 
\begin{itemize}
\item We won't obtain a severe understanding of the \gls{Tygron} engine.
\item We won't obtain knowledge about other programming languages then the ones used in the plugin.
\end{itemize}

\subsection{Initial release plan}
Our project is divided in three separate parts. We started by finding out what was expected from us and how we could start at working on the final product (getting an understanding of the \gls{GOAL} programming language code). Then we ported \gls{GAMYGDALA} to Java. The final and biggest part of this project is de\gls{SIG}ning and implementing the final plugin using the previously acquired knowledge. \\ \par
There are two major milestones in our release plan. These are the initial input by \gls{SIG} in week 6, and the final input by \gls{SIG} in week 9. \par 
In week 6, we wan't to have a working, refactored and tested \gls{GAMYGDALA} port that follows the \gls{SEM}. We also want to have our first version of the \gls{GOAL} plugin. It needs to have a decent test coverage and it had to follow the \gls{SEM}. \par
We will use the last weeks to process the feedback given by \gls{SIG} and improving the overall code and test coverage percentage. The last weeks will also be used to work on the documentation of the project as well as the documentation of the code. The final version will be delivered in week 9.