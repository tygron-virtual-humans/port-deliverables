\section{Product Backlog}
In this section you can find the product backlog for our part of the project. It is made using the MoSCoW method.

\subsection{User stories of features}
\paragraph{Must have} 
\begin{itemize}
\item There needs to be a port gamygdala
\item implemented in goal
\item koppeling gamygdala goal
\item work without big problems
\end{itemize}

\paragraph{Should have} 
\begin{itemize}
\item fast
\item complete gamygdala work
\end{itemize}

\paragraph{Could have} 
\begin{itemize}
\item make plugin framework goal
\end{itemize}

\paragraph{Won't have} 
\begin{itemize}
\item emotions on init
\item automatic emotion recognition
\end{itemize}

\subsection{User stories of technical improvements}
\paragraph{Must have} 
\begin{itemize}
\item Sem principals
\item 70\% getest
\item unit / integration
\end{itemize}

\paragraph{Should have} 
\begin{itemize}
\item javadoc
\item checkstyle
\end{itemize}

\paragraph{Could have} 
\begin{itemize}
\item higher test scores
\end{itemize}

\paragraph{Won't have} 
\begin{itemize}
\item improve goal framework
\item change gamygdala work
\end{itemize}

\subsection{User stories of know-how acquisition}
\paragraph{Must have} 
\begin{itemize}
\item Understanding of goal
\item understanding of gamygdala
\item scrum skills
\item infovaardigheden
\item interactive design
\end{itemize}

\paragraph{Should have} 
\begin{itemize}
\item
\end{itemize}

\paragraph{Could have} 
\begin{itemize}
\item
\end{itemize}

\paragraph{Won't have} 
\begin{itemize}
\item
\end{itemize}

\subsection{Initial release plan}
Our project is divided in three separate parts. We started by finding out what was expected from us and how we could start at working on the final product (getting an understanding of the GOAL programming language code). Then we ported GAMYGDALA to Java. The final and biggest part of this project is designing and implementing the final plugin using the previously acquired knowledge.