\section{Product}
\subsection{High-level Product Backlog}
We as a group are part of the bigger context project, which consists of four groups that all work on a different part of the final product. Our part of this final product is making a GAMYGDALA plugin for the GOAL programming language. To achieve this, we first need to get a good understanding of the GOAL programming language. Then we need to find out how we can make a (simple) plugin for GOAL. We also need to have a certain understanding of how GAMYGDALA works, so we can port GAMYGDALA to Java. Then we need to bring it all together so we can create a working GAMYGDALA plugin for GOAL. \\ \par
Our final plugin will be used by the other groups to make a virtual human for the Tygron engine. You can see the Product View for an elaborate description of our user and the final product.

\subsection{Roadmap}
In this subsection, you can find a global planning per sprint/week.

\paragraph{Sprint 1} The first week was used to start up the project. We followed seminars about the context project in general and about our specific context project. We also divided the groups within our project. We found out what was expected of us during this project.
\paragraph{Sprint 2} Sprint 2 was used to gain some understanding for the way that the GOAL programming language works 'under the hood'. We did this by implementing a simple calculator plugin for GOAL.
\paragraph{Sprint 3} This sprint will be used to port GAMYGDALA to Java. To do this, we first need a global understanding of how GAMYGDALA works. We then need to port GAMYGDALA, which is written in Javascript, to Java. Finally, we need to refactor the Java code so it follows the Software Engineering principals.
\paragraph{Sprint 4} The fourth sprint will be used to make an action/percept base integration of GAMYGDALA into the GOAL programming language. This will be similar to the Javascript interface, but in the action/percept interface of GOAL.
\paragraph{Sprint 5} Sprint 5 will be used to make a simple appraisal module in the GOAL programming language which is based on actions/percepts.
\paragraph{Sprint 6 to 9} These sprints will be used to make the full GAMYGDALA appraisal module. These sprints will also be used to work on the final project report. Furthermore, we can also use these sprints to finalize tasks that weren't finished during the previous sprints.