\section{Overview}
The final product is a GAMYGDALA \citep{gamygdala} plug-in for the GOAL programming language \citep{goal}. The final product can be split into two separate parts: the GAMYGDALA port and the GOAL plug-in.

\subsection{GAMYGDALA port}
``GAMYGDALA is an emotional appraisal engine that enables game developers to easily add emotions to their Non-Player Characters (NPC)." \citep{gamygdala}. \par
GAMYGDALA is written in Javascript, but for it to properly work with GOAL, which is written in Java, there was a need to port GAMYGDALA to Java. This GAMYGDALA port is the biggest software product of this project. When the initial port was finished, we noticed that the Java code did not really follow the Software Engineering principles. This is why we decided to do a total code refactor. \par 
The final version of the port is far better then the original port on account of the Software Engineering principles and it behaves in the same way as the original GAMYGDALA code. You can read more about this refactor in the Emergent Architecture Design \citep{ead} and in next chapter.

\subsection{GOAL plug-in}
The other part of the software product is the GOAL plug-in. The GOAL source code had to be altered to enable the usage of GAMYGDALA in GOAL. There now is a fully working plug-in that developers can use in their GOAL programs. The full GAMYGDALA functionality can be used within the GOAL environment.