\section{Evaluation}
In this section you can read our evaluation about our experiences during this Contextproject. It consist of an evaluation of the product, a failure analysis, an evaluation of collaboration between our group members as well as an evaluation of the collaboration between the separate groups of our Contextprojext.

\subsection{Product}
This subsection is divided into two parts, the GAMYGDALA port and the GOAL plug-in.
\subsubsection{GAMYGDALA port}
This was the first project in which we had to work with code that we did not make ourselves. We had to work with the GAMYGDALA engine that was already made, this was an entirely new experience. The initial idea was to literally port the GAMYGDALA Javascript code to Java, this was not a lot of work. When we finished this initial port, we found out that it did not really follow the Software Engineering principles. Because our Software Engineering skills are also rated during this project, we decided to start refactoring the port. You can read more about the problems that we encountered during this refactor in the Failure analysis subsection. \par
Now that the port is finished, we are quite happy we the result. Although the code is not perfect yet, it is a lot better then the port that we started with. It now follows the Software Engineering principles a lot better, it uses design patterns and for example the god classes are eliminated now. You can read more about this in our Emergent Architecture Design \citep{ead}. \par
It was a great experience to work on someone else's code because we encountered problems that we did not encounter in previous projects. We learned a  lot of new skills that will be useful in the future.
\subsubsection{GOAL plug-in}
A lot of what we wrote about the GAMYGDALA port also holds for our GOAL plug-in. We spent a lot of time in understanding how GOAL works `under the hood'. We needed to truly get how GOAL works before we could start making a plug-in for it. Over time we gained more knowledge about GOAL and we started making a simple calculator plug-in. This took more time then we first anticipated, but we learned a lot from making it. This knowledge was used later on to make the actual plug-in for GAMYGDALA. \par
Though we also worked on code that we did not write ourselves, there are also some differences between making the GAMYGDALA port and this plug-in. In the port code we worked on every class, but in the GOAL code we only added and changed code in certain classes. Making the plug-in was more about adding code and functionality and the port was more about copying the code and functionality. It was very nice to have these two experiences within one project.

\subsection{Failure analysis}

\subsection{Collaboration between our group members}

\subsection{Collaboration between the groups}