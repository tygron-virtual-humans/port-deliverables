\section{Evaluation}
In this section you can read our evaluation about our experiences during this Contextproject. It consist of an evaluation of the product, a failure analysis, an evaluation of collaboration between our group members as well as an evaluation of the collaboration between the separate groups of our Contextprojext.

\subsection{Product}
This subsection is divided into two parts, the GAMYGDALA port and the GOAL plug-in.
\subsubsection{GAMYGDALA port}
This was the first project in which we had to work with code that we did not make ourselves. We had to work with the GAMYGDALA engine that was already made, this was an entirely new experience. The initial idea was to literally port the GAMYGDALA Javascript code to Java, this was not a lot of work. When we finished this initial port, we found out that it did not really follow the Software Engineering principles. Because our Software Engineering skills are also rated during this project, we decided to start refactoring the port. You can read more about the problems that we encountered during this refactor in the Failure analysis subsection. \par
Now that the port is finished, we are quite happy we the result. Although the code is not perfect yet, it is a lot better then the port that we started with. It now follows the Software Engineering principles a lot better, it uses design patterns and for example the god classes are eliminated now. You can read more about this in our Emergent Architecture Design \citep{ead}. \par
It was a great experience to work on someone else's code because we encountered problems that we did not encounter in previous projects. We learned a  lot of new skills that will be useful in the future.
\subsubsection{GOAL plug-in}
A lot of what we wrote about the GAMYGDALA port also holds for our GOAL plug-in. We spent a lot of time in understanding how GOAL works `under the hood'. We needed to truly get how GOAL works before we could start making a plug-in for it. Over time we gained more knowledge about GOAL and we started making a simple calculator plug-in. This took more time then we first anticipated, but we learned a lot from making it. This knowledge was used later on to make the actual plug-in for GAMYGDALA. \par
Though we also worked on code that we did not write ourselves, there are also some differences between making the GAMYGDALA port and this plug-in. In the port code we worked on every class, but in the GOAL code we only added and changed code in certain classes. Making the plug-in was more about adding code and functionality and the port was more about copying the code and functionality. It was very nice to have these two experiences within one project.

\subsection{Failure analysis}
Not everything went perfectly according to plan during this project. In this subsection you can read about the biggest problems that we encountered. \par 
In the beginning of the project, our biggest focus was on gaining knowledge about how GOAL works. Here we also encountered our first problem. Until that point we did not have much experience in working with complex code from other people. We might have underestimated the time it would take to really understand what was going on in GOAL. This is why the calculator plug-in that was planned to take about a week actually took us a lot longer. \par
After the first weeks of the project we also started porting GAMYGDALA. The initial version of the port did not give us to many problems, it just took a lot of time to work trough all the Javascript code. The first big problem that we encountered was when we ran InCode Helium \citep{incode} over this first version. The results were quite bad from a software engineering perspective. As stated above in the report, this problem was solved by an extensive refactor of the entire code. \par
Another problem that we encountered during this project was that is very hard to plan the right tasks every week. A great advantage of this Contextproject over the others was that we did not have a tight schedule in which every step was clear from week 1. Although we really liked the reality side of this project, it made it also a lot harder to foresee which tasks would take a lot of time during each week. We just had to plan in any task that we could think of in the beginning and then see what came up during the week. \par
An ongoing problem during the refactoring was that if you change a certain class, all of the classes and tests that depend on it also need to be changed. This often led to a `domino' effect of problems, solving one problem led to another which also led to another and so on.

\subsection{Collaboration between our group members}
Our group consists of five members. During the project we decided to involve everybody with all the different parts of the project as much as we could. This way everybody learnt something about all the different aspects of our product. \par
We did have a distribution of the responsibilities amongst the team members. This means that for example Wouter spend more time working on the GOAL plug-in and Sven H spend more time working on the port. We did this to make sure that every task was would be finished. \par
We did not encounter problems within our group during this project. We had clear rules that every team member followed and we all worked equally hard on this project. We are very happy with our group process during this project.

\subsection{Collaboration between the groups}
Our Contextproject differs from the other Contextprojects because the groups within our Contextproject are all working together on one product, instead of all the groups working on their own product. \par
This is why the collaboration between the different groups was also a huge part of this project. We worked together on code as well as on reports. The communication between the groups was great, we could ask each other questions and help each other with the project. We really liked this part of the project because it gives a realistic view on actual jobs that include working on a software project with different people and groups.