\section{Outlook}
In this section we will give an outlook on where our product can be used and on the future improvements to our product.

\subsection{Usage of our product}
In this project, our product will be used by one of the other groups to create a virtual human for the Tygron Engine. However, this is not the only field in which our product can be used. Our plug-in supports the full GAMYGDALA functionality, this means that it can be used in almost every GOAL project. \par
For example, our product can be used in the first year Multi Agent Systems project, were students get the opportunity to program a team of bots in the Unreal Tournament environment \citep{UT}. The bad feelings for an opponent bot can grow worse every time it scores a point. This way your own bots can focus more on the bots that they hate more which are the best bots on the other team (the bots that are scoring the most points).

\subsection{Improvements to our product}
Our final product is not perfect yet. The time we had for the entire project was ten weeks, which is not enough to make a perfect product. The functionality of our port and plug-in passes all our tests and seems to be working like it should, but we do not yet have a 100\% test coverage. This might be impossible to achieve, but our test coverage can be approved. \par
Another part that we can improve is the refactor. Refactoring, and making sure that all the functionality is kept the same, takes a lot of time. We managed to fix many design flaws that were in our initial port, but there are still some parts that could be improved. \par
The documentation could also be extended, we could create examples and assignments in which you could train yourself in using the plug-in.