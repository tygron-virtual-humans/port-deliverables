\section{Interaction Design}
This section contains the interaction design report. It includes information about the target audience of the final product, how these people have used the product and what can be learned from the test results of these interactions.

\subsection{Users}
The final product is a fully functioning version of GAMYGDALA in the GOAL programming environment. GOAL programmers who want to use emotions for their GOAL agents are the target audience of the product.

To properly test user interaction with the GOAL plug-in, testers with appropriate background knowledge of the GOAL programming language had to be found. Because the GOAL is a language that is taught in the first year of the Computer Science bachelor at TU Delft during the courses Logic Based Artificial Intelligence and Multi Agent Systems project, there are a lot of Computer Science students that are familiar with GOAL.

Second year Computer Science students have been asked to try out the GAMYGDALA plug-in for GOAL. Two groups of students have been selected: students who are also working on the Virtual Humans for Serious Gaming project, and students who work on other Contextproject subjects. This way, opinions of students who already know about the product and how it should work have been collected, as well as opinions from students that have no prior knowledge of our product.

\subsection{Test method}
User interaction testing is all about users actually using our product, writing GOAL code, and taking opinions on usability and user-friendliness. Because code writing is fairly hard to measure in a meaningful way, the \textit{thinking-aloud} method \citep{thinking-aloud} has been used to evaluate user interaction. Testers have been asked to say anything that came up in their minds while using the product. Positive and negative emotions could therefore easily be distinguished. Users were encouraged to express their feelings about all aspects of the product including overall usability, but also about missing features, naming convention and other facets. All remarks were noted and the results are discussed in paragraph \ref{subsec:iadres}.

The test set-up consisted of a laptop with a working version of SimpleIDE and a BlocksWorld for Teams\footnote{\url{https://github.com/eishub/BW4T}} instance running on a GOAL version extended with GAMYGDALA. Students of Computer Science, familiar with GOAL, were asked to evaluate a sample program and write with additional code for it afterwards. A manual describing GOAL actions, a short GAMYGDALA introduction and a list of all emotions in GAMYGDALA was provided. \clearpage

\subsection{Test results}
\label{subsec:iadres}
This section contains the test results. Firstly, the overall reaction on the product will be reviewed. Secondly, the positive feedback about the GAMYGDALA plug-in is examined and finally the user remarks regarding possible improvements are discussed.

\subsubsection{Overall reaction}
The overall reaction on the product was very positive. At first, however, certain aspects of GAMYGDALA, such as the inner workings with goals, beliefs and likelihoods, were unclear for the users. After clarification, most of the confusion was resolved. From this, however, can be concluded that a more encompassing introduction to GAMYGDALA is desirable.

Users were immediately interested in GAMYGDALA after being fully instructed about the possibilities. They already started to imagine what it could mean for computer games if GAMYGDALA was used in it. Because of their interest, it was very easy to discuss certain topics of the product with the users. They gave very clear and targeted feedback.  Most of it was positive, as discussed in section \ref{subsubsec:iadposfeedb}, but some feedback was about possible improvements, which is explained in section \ref{subsubsec:iadposimprov}. The overall conclusion is that GAMYGDALA is very promising, but a user would need a thorough understanding of the engine to really get the most out of it.

\subsubsection{Positive feedback}
\label{subsubsec:iadposfeedb}
The positive feedback was generally based on the purpose of GAMYGDALA and the possibilities it opens for the (serious) gaming industry. Relations and emotions are fundamental elements of life which computers are not capable of simulating, and GAMYGDALA was generally seen as a step in the right direction by the users. Some people directly mentioned the positive addition of a more human touch of non-player characters in-game, would emotions be added.\\

\hspace{-\parindent}Additional positive feedback included the following.
\begin{itemize}
\item The GAMYGDALA manual is clear for the users. There were few questions about the purpose and content.
\item The percepts about emotions returned by GAMYGDALA are clear and concise. 
\item The logic behind GAMYGDALA, the emotion generation, is understandable. The emotions resulting from particular actions are as expected in that situation.
\item No visible delay was observed while using GAMYGDALA, indicating all necessary calculations are performed quickly.
\end{itemize}

\subsubsection{Possible improvements}
\label{subsubsec:iadposimprov}
With the feedback from the users came a few points for improvement.

Firstly, the naming of actions and variables was unclear. For example, users did not intuitively understand the meaning of the action ``ga-appraise''. ``Appraise'' might be the wrong choice of words, as for example ``ga-event'' or ``ga-value'' are more meaningful action names. Furthermore, the ``isMaintenanceGoal'' variable, which determines whether or not an achieved goal is to be removed from the goal base, was hard to understand by users.

Being the project team which has worked for weeks with the code, the meaning of above action and variable is clear. For new users, however, these names are not descriptive at all. Whether or not this is to be changed in a future version is up for debate, because the working of all actions, methods and variables can be looked up in the manual. It is all the same a good idea to evaluate the naming of unintuitive actions, methods and variables.

Another point of confusion among users was the \textit{likelihood} of goals and beliefs. When appraising an event, the belief as well as the affected goals have a \textit{likelihood}, but each with a different meaning. The difference of these is not clear from the manual or the variable names. Explaining the difference proved to be rather difficult, so a clear explanation in the manual is required.\\

\hspace{-\parindent}Finally, some points of improvement about the working of GAMYGDALA itself were mentioned.
\begin{itemize}
\item The sensitivity of agents for certain emotions is currently not adjustable. In real life, some people are more affected by positive emotions than by negative emotions, and vice versa.
\item The intensity of a relation between two agents in a GAMYGDALA environment is fixed, and does not change based on positive of negative actions by the causal agent of the relation. If another agent consistently undermines your own goals, the intensity of the relation should decrease.
\end{itemize}