\section{Interaction Design}
This section contains our interaction design report. It explains who will be the users of the final product, how these users tested the product and what we have learned from the results of these tests.

\subsection{Users}
As stated before, our final product is a fully functioning version of GAMYGDALA in the GOAL programming environment. Therefore, our final users will be GOAL programmers who want to use emotions for their GOAL agents. \par
To find testers we looked for students who were familiar with the GOAL programming language. Because the GOAL programming language is a language that is taught during the first year of the Computer Science bachelor (Logic Based Artificial Intelligence and the Multi Agent Systems project) there are a lot of Computer Science students that are familiar with GOAL. \par
We decided to ask second year Computer Science bachelor students to help us by trying to use our GAMYGDALA plug-in in GOAL. We chose some students who are also working on our Virtual Humans for Serious Gaming Contextproject, as well as some students who work on other Contextproject. This way we could collect the opinions of students who already know about our product and how it should work, as well as the opinions from students that did not use our product before.

\subsection{How do we test}
The best way to let our testers use our product is by just letting them play with it. This is why we asked them to try to give an agent some simple emotions. \par
For this test we used the Thinking-aloud method \citep{thinking-aloud}. This means that we asked our testers to say anything that came up in their mind during the test. This was the easiest way to find out if the testers enjoyed using our product and if the usage was easy enough.\par
The set-up of the test was based on the way that students of computer science got taught GOAL in the first place. There was a GAMYGDALA manual including; a overview of GAMYGDALA, a list of GAMYGDALA actions in GOAL and a list of the emotion's which could be returned. Besides the manual there was a test environment built on top of the BlockWorld example project of GOAL. The test environment included the full GAMYGDALA plug-in.

\pagebreak

\subsection{Results}
The results of the test will be divided into a few different points. We will start with the general reaction on the system. Then we will discuss the positive feedback about the GAMYGDALA plug-in and at last we will discuss the improvements which could be made based on the users' feedback.

\subsubsection{General reaction}
The overall reaction on the product was very positive. Everyone has emotions and understands basic emotions. Those maybe two of the reason that when the users started reading the overview of GAMYGDALA they were instantly interested. Many of them already started to imagine what it could mean for computer games if GAMYGDALA was applied to it. Because of the interest of the users, it was very easy for us to discuss certain topics of the systems with the testers. They gave very clear and targeted feedback on certain elements of the system. Most of the feedback was positive, it was very clear for users what GAMYGDALA could do and what the emotions were that they got back. The improvement points were mostly based on haziness of variable names, action names or what the variables exactly meant within the engine. The general conclusion was that GAMYGDALA is very cool, but there needs to be more clearity and a basic understanding of the engine to really get the best out of it.

\subsubsection{Positive feedback}
The positive feedback was generally based on the purpose which GAMYGDALA could have in the (serious) gaming industry. Relations and emotions are really basic elements of life which are not yet fully implemented into (serious) games. Some people directly mentioned the more human taste of non-player characters in-game, if emotions were added to their profile. The GAMYGDALA manual was clear for the users, they did not have questions about what the engine could do. One of the more remarkable reaction about relations was: 'Oh nice, you can even make a relation with non existing persons'. What also gave a positive reaction was the emotion percepts that where given back. It was very clear to the users what the emotions meant and why they were generated by the engine.

\subsubsection{Points for improvement}
With the feedback from the users came a few points for improvement. A lot of these points were about the naming of actions and variables. The names of these actions and variables, like the 'ga-appraise' action and the 'isMaintenanceGoal' variable, came directly from the original code. Subsequently this was a clear point of improvement. This was because the project team knew these names from working with the system, but for users this is naturally not obvious what they did. The 'ga-appraise' action could be altered to something like 'ga-belief', and the 'isMaintenanceGoal' variable could be changed to something like 'repeatable'. A big point of confusion among users was the likelihood of goals and appraises. The two likelihoods have a different meaning, but this is not clear from the manual or the variable names. After some explanation the users understood the meaning of these variables, but it needs to be clear in the manual and naming. At last we got some points of improvement, not especially for our plug-in, but for the overall GAMYGDALA engine. One of the points was the sensitivity of agents for certain emotions. In real life some people are more affected by positive emotions then by negative emotions or vice versa. Another point was group relation, an overall group state could affect personal emotions. These two points could be implemented in GAMYGDALA in the future.