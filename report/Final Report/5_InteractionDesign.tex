\section{Interaction Design}
This section contains the interaction design report. It explains who will be the users of the final product, how these users tested the product and what is learned from the results of these tests.

\subsection{Users}
The final product is a fully functioning version of GAMYGDALA in the GOAL programming environment. GOAL programmers who want to use emotions for their GOAL agents are the target audience of the product.

To properly test user interaction with the GOAL plug-in, testers had to be found with appropriate background knowledge of the GOAL programming language. Because the GOAL is a language that is taught in the first year of the Computer Science bachelor, during the courses Logic Based Artificial Intelligence and Multi Agent Systems project, there are a lot of Computer Science students that are familiar with GOAL.

We decided to ask second year Computer Science bachelor students to help us by trying to use our GAMYGDALA plug-in in GOAL. We chose some students who are also working on our Virtual Humans for Serious Gaming Contextproject, as well as some students who work on other Contextproject. This way we could collect the opinions of students who already know about our product and how it should work, as well as the opinions from students that did not use our product before.

\subsection{How do we test}
The best way to let our testers use our product is by just letting them play with it. This is why we asked them to try to give an agent some simple emotions. \par
For this test we used the Thinking-aloud method \citep{thinking-aloud}. This means that we asked our testers to say anything that came up in their mind during the test. This was the easiest way to find out if the testers enjoyed using our product and if the usage was easy enough.

\subsection{Results}
The results will be discussed here in the final report. This draft does not contain these results yet.