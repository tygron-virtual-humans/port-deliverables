\section{Interaction Design}
This section contains our interaction design report. You can read who will be the users of our final product, how we let these users test our product and what we learned from the results of these tests.

\subsection{Users}
As stated before, our final product is a fully functioning version of GAMYGDALA in the GOAL programming environment. Therefore, our final users will be GOAL programmers who want to use emotions for their GOAL agents. \par
To find testers we looked for students who were familiar with the GOAL programming language. Because the GOAL programming language is a language that is taught during the first year of the Computer Science bachelor (Logic Based Artificial Intelligence and the Multi Agent Systems project) there are a lot of Computer Science students that are familiar with GOAL. \par
We decided to ask second year Computer Science bachelor students to help us by trying to use our GAMYGDALA plug-in in GOAL. We chose some students who are also working on our Virtual Humans for Serious Gaming Contextproject, as well as some students who work on other Contextproject. This way we could collect the opinions of students who already know about our product and how it should work, as well as the opinions from students that did not use our product before.

\subsection{How do we test}
The best way to let our testers use our product is by just letting them play with it. This is why we asked them to try to give an agent some simple emotions. \par
For this test we used the Thinking-aloud method \citep{thinking-aloud}. This means that we asked our testers to say anything that came up in their mind during the test. This was the easiest way to find out if the testers enjoyed using our product and if the usage was easy enough.

\subsection{Results}
The results will be discussed here in the final report. This draft does not contain these results yet.