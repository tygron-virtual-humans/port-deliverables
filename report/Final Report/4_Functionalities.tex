\section{Developed functionalities}
It is essential for Non-Playable Characters (NPC) in games to have and feel emotions. Emotions are crucial in respect to gameplay decisions, especially in serious games. The virtual human to which we contribute with this project has to be able be affected by actions outside it's own influence, and feel emotions. Therefore, we have tried to find a way to use GAMYGDALA \citep{gamygdala} in GOAL.

\subsection{Gamygdala Java Port}
GAMYGDALA is an emotion engine for games originally written in JavaScript. A large part of the work done during this project has been porting the JavaScript version to Java while simultaneously optimizing code quality. While we have not developed GAMYGDALA ourselves, we will briefly discuss its functionality for completeness.

GAMYGDALA operates in an environment where actors, ''agents'', interact with each other. Every agent has goals to achieve. In the environment, events can occur which influence the likelihood of goal achievement for a particular agent. Taken into account all the events that have happened, one or more emotions are deduced from the agents internal emotional state. Furthermore, agents have reciprocal relations: agents ''feel'' for other agents and adjust their emotions accordingly.

\subsection{Gamygdala in GOAL}