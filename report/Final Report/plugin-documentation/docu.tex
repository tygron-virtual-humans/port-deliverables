\section{Appendix: Plug-in Manual}
\subsection{Introduction}
This is the manual for the GAMYGDALA extension of the GOAL programming language. Before reading this manual, you will have to understand how the GOAL language itself works. If you still need to read up on this, it is suggested that you read the GOAL programming manual and try out some of the exercises.

To offer you a full understanding of how the plug-in in works, you will find in this document an overview of how GAMYGDALA works, definition of all actions that can be used, how to use them, and the results that are yielded when using the plugin.

\subsection{Gamygdala overview}
In GAMYGDALA, an agent gains emotions by adopting goals and appraising beliefs that influence the likelihoods of these goals. For example, if an agent has a goal to become rich, and he signs a large deal, its likelihood to become rich will become higher, and the agent will become hopeful. As in the real world, the intensity of these emotions decays over time.

Agents may also have relations; they can like or hate each other to a certain extent. These relations will inflict emotions such as happy-for or resentment when something happens to the agents that they feel liking or hate towards. For example, if Mario kills a Goomba, Peach may start feeling happy-for and bowser may start feeling resentment.

\subsection{Actions}
The GAMYGDALA plugin for goal is based on new actions. These actions are very similar to the environment actions that are generally defined in the actionspec. No actionspec is needed for this plugin, though.
This section covers the actions that can be used. The next section will cover the results that these actions yield.

\subsubsection{ga-adopt}

\textbf{ga-adopt}(term \emph{goal}, double \emph{likelihood}, boolean \emph{isMaintenanceGoal})

Adopts a goal in the GAMYGDALA engine.

\begin{tabular}{l l}
	\textbf{goal} & The goal that is being adopted.\\
	\textbf{likelihood} & The likelihood of the goal being achieved,\\
	& from 0 to 1.\\
	\textbf{isMaintenanceGoal} & Whether or not the goal can be achieved \\
	& multiple times during the session.\\
\end{tabular}
\clearpage
\subsubsection{ga-drop}

\textbf{ga-drop}(term \emph{goal})

Drops a goal in the GAMYGDALA engine. \\

\begin{tabular}{l l}
	\textbf{goal} & The goal that is being adopted.\\
\end{tabular}
\subsubsection{ga-create-relation}

\textbf{ga-create-relation}(string \emph{otherAgent}, double \emph{relation})\vspace{.5em}

\hspace{-\parindent}Creates a relation towards another agent. This can already be called when the other agent does not `exist' yet. \\[.4em]

\noindent
\begin{tabularx}{\textwidth}{@{}p{4cm} X}
	\textbf{otherAgent} & The name of the other agent.\\
	\textbf{relation} & The intensity of the relations, from -1 to 1, where -1 is full hate, and 1 is full love.\\
\end{tabularx}
\subsubsection{ga-appraise}

\textbf{ga-appraise}(double \emph{likelihood}, string \emph{agent}, list \emph{affectedGoals},\\list \emph{goalCongruences}, boolean \emph{isIncremental})\vspace{.5em}

\hspace{-\parindent}Appraises a belief that influences the likelihoods of goals. This is how agents gain emotions.\\[.4em]

\noindent
\begin{tabularx}{\textwidth}{@{}p{4cm} X}
	\textbf{likelihood} & The likelihood of the belief to be true.\\
	\textbf{agent} & The agent that caused the belief. \\
	\textbf{affectedGoals} & The goals that were affected.\\
	\textbf{goalCongruences} & How the goals were affected respectively, from -1 to 1, -1 for very negatively, 1 for very positively. \\
	\textbf{likelihood} & The likelihood of the belief to be true.\\
	\textbf{isIncremental} & Whether this belief enforces the likelihood of these goals (\textit{true}), or defines the absolute likelihood (\textit{false}). 
\end{tabularx}
\subsection{ga-decay}

\textbf{ga-drop}(any \emph{any})

Decays all emotions and updates them in the belief base.

\begin{itemize}
	\item \textbf{any} \\ Any parameter. This does not have any influence and will be removed in the future.
\end{itemize}




\clearpage
\subsection{Results}
When the appraise and decay actions are called, the emotions for the agent are fetched from GAMYGDALA, and they are stored in the belief base, in the following form:
\begin{center}
	\texttt{emotion(EmotionName,Intensity)}
\end{center}

\hspace{-\parindent}Basic emotions can one of the following:\\

\noindent
\begin{tabular}[H]{@{}p{2.59cm} l}
	Hope & Fear \\
	Satisfaction & Fear-Confirmed \\
	Joy & Distress \\
	Disappointment & Relief
\end{tabular}

\vspace{1em}\hspace{-\parindent}And relational emotions can be the following:\\

\noindent
\begin{tabular}[H]{@{}p{2.59cm} l}
	Happy-for & Resentment \\
	Pity & Gloating \\
	Gratitude & Anger \\
	Gratification & Remorse
\end{tabular}

\vspace{1em}\hspace{-\parindent}The intensity of emotions ranges from 0 to 1.